\chapter{Testen}%Siebe
\subsubsection{Testmogelijkheden}
%Een overzicht van de testen en testmogelijkheden die werden ingebouwd of
%uitgevoerd.
Tijdens het ontwikkelen van een programma of applicatie is testen essentieel. Een goede gewoonte is om na het maken van een component deze grondig te testen en zoveel mogelijk fouten trachten te verwijderen.
Door gebruik te maken van de uitgebreide debug-mogelijkheden van Android Studio en het bijhouden van logs konden problemen vaak snel worden opgespoord. De moeilijkheid in dit project was het werken met Google App Engine. De uitvoering van de backend-code gebeurt op de servers van Google en kan dus niet worden gedebugged. Gelukkig kan men wel logs laten genereren en deze na uitvoering bekijken.
Zo konden fouten gelokaliseerd en opgelost worden.
Om de Android applicatie en webinterface voor het grote publiek open te stellen is het echter noodzakelijk om nog verschillende testrondes uit te voeren. Hierbij wordt de laatste recente versie van de Android toepassing verspreid via de website. Om gemakkelijk feedback van gebruikers te verwerken wordt een feedbackmechanisme in de applicatie geïntegreerd. Dit wordt verder toegelicht in de volgende sectie.

\subsubsection{Feedback}
% Backend testen, statistieken over de endpoints
% Userfeedback: Integratie van feedbacksysteem om gebruikervaringen bij te houden.
Een applicatie is vrijwel nooit foutvrij, noch voldoet ze volledig aan de eisen van de gebruiker. Door standaard de mogelijkheid tot feedback aan te bieden in de toepassing, wordt gebruikers de mogelijkheid geboden om suggesties, problemen en fouten door te geven.
In de Android applicatie gebeurt dit vanuit de toepassing zelf, en wordt de gebruiker via notificaties aangespoord om feedback te geven. Feedback van de gebruikers wordt momenteel doorgestuurd naar het emailadres van Triump. Vanhieruit kunnen de groepsleden de feedback raadplegen.

