
\chapter{Besluit}%Jonas

% Een besluit.
% Wat hebben wij ons aangedaan? Konnen we niet gewoon een website maken?
In dit vakoverschrijdend project werd een applicatie ontwikkeld die voortbouwd op de functionaliteit van Foursquare en deze uitbreidt.
Met Triump kunnen gebruikers zich verenigen in groepen en in competitie treden met andere groepen via de check-in functionaliteit van Foursquare.
Verder kunnen gebruikers evenementen aanmaken, waaraan een beloning kan hangen. Zo kunnen uitbaters van bepaalde locaties Triump gebruiken om klanten te lokken.
Triump bestaat uit 3 grote delen.\\
Er is de android applicatie, deze laat toe om, naast aanmaken van groepen, bekijken van evenementen,... , in te checken in een locatie.
Het resultaat voldoet aan de specificaties die in het begin werden gesteld en is zo goed als klaar om door personen te worden gebruikt.\\
De data (groepen, evenementen en dergelijke) die door de applicatie wordt gebruikt, bevindt in de centrale backend van Triump.
Voor deze backend werd gebruik gemaakt van App Engine, een dienst aangeboden door Google.
Het opstellen van de backend gebeurde met schaalbaarheid en snelheid in gedachte, zodat Triump ook met een groter aantal gebruikers kan omgaan.\\
Om aan de noden van uitbaters van locaties te voorzien, werd ook een webinterface voorzien. 
Hiermee kan men een locatie registeren en officiële evenementen organiseren voor geregistreerde locaties.\\

Deze 12 weken waren voor alle leden een leerrijke ervaring. We hebben veel bijgeleerd, zowel zeer specifieke zaken, zoals Android development, als meer algemene competenties, zoals project planning.\\
% Het toepassen van kennis die we tijdens de voorbije drie jaar hebben opgedaan gaf veel voldoening.
We kunnen besluiten dat Triump ... % add your conclusions