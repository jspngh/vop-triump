
\chapter{Besluit}%Jonas

In dit vakoverschrijdend project werd een applicatie ontwikkeld, gericht op het uitbreiden van de functionaliteit van Foursquare en Qustomer.
Met Triump kunnen gebruikers zich verenigen in groepen en in competitie treden met andere groepen via de checkin functionaliteit van Foursquare.
Verder kunnen gebruikers evenementen aanmaken, waaraan een promotie kan gekoppeld worden. Zo kunnen uitbaters van bepaalde locaties Triump gebruiken om aan online marketing te doen.\\


Triump bestaat uit drie grote delen.
Er is de Android applicatie voor de traditionele gebruiker. Hiermee kan men groepen aanmaken, lid worden van groepen, evenementen bekijken en inchecken in een locatie.
De ontwikkelde proof of concept voldoet aan de specificaties die bij aanvang van het project werden vooropgesteld. Alvorens Triump te kunnen lanceren in de Google App Store is echter het belangrijk om nog verschillende fases met testgebruikers door te lopen.
De data (groepen, evenementen en dergelijke) die door de applicatie wordt gebruikt en de functies die de applicatie oproept, bevinden zich in de centrale backend van Triump.
Voor deze backend werd gebruik gemaakt van App Engine, een dienst aangeboden door Google.
Het opstellen van de backend gebeurde met schaalbaarheid en snelheid in gedachte, zodat Triump ook met een groter aantal gebruikers kan omgaan.
Om aan de noden van uitbaters van locaties te voorzien, werd een webinterface voorzien. 
Hiermee kan men een locatie registeren en officiële evenementen organiseren voor geregistreerde locaties.\\


Deze 12 weken waren voor alle leden een leerrijke ervaring. Zowel zeer specifieke zaken, zoals Android development, als meer algemene competenties, zoals project planning kwamen aan bod.
We kunnen besluiten dat Triump, in het bijzonder voor de vier leden, een boeiend project was. Hierbij werd kennis gebruikt, opgedaan in afgelopen drie bachelorjaren, en was er bovendien veel ruimte om nieuwe vaardigheden bij te leren.
Tenslotte bedanken we hierbij ook onze begeleiders Tim Verbelen en Bert Vankeirsbilck.