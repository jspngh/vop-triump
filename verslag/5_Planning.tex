
\chapter{Planning en taakverdeling}

% De initieel opgemaakte planning en de uiteindelijke uitvoering (in Ganttchart-achtige
% vorm) met een korte verklaring van de verschillen.

\subsubsection{Initiele planning}

Figuur \ref{fig:planning_initieel} in de bijlagen toont de initiële planning. De planning werd opgesteld naar aanleiding van de eerste presentatie en is bijgevolg relatief vaag. Desalniettemin is het hoofdprincipe van deze initiële planning doorheen het volledige project blijven gelden. 

Het prinicipe weergegeven in figuur \ref{fig:planning_initieel} is de gelaagde opbouw van Triump. Er wordt namelijk met drie "cores" gewerkt. De eerste core van de drie cores bevat de absolute basis van de applicatie zoals communicatie met de backend, een basis-activiteit die de locatie van een gebruiker bepaald en een login en logout scherm. In de tweede core worden de basisonderdelen van Triump zoals locaties, events, rewards, groepen, gebruikers en een basis UI geïntegreerd in de applicatie. Een laatste core richt zich voornamelijk op gebruiksvriendelijkheid en afwerking. In deze core bevinden zich elementen zoals een notificatiesysteem, instellingenpagina, feedbacksysteem en een aantrijkelijke UI. Het was de bedoeling de core gelijktijdig op de backend en de frontend te implementeren.

Na het uitwerken van de kern werd tijd voorzien voor uitbreidingen. Mogelijk uitbreidingen waren: het voorzien korte spelletjes om op een tweede manier punten te kunnen verdienen, iOS of WindowsPhone versie en een integratie met Qustomer. 

De laatste weken van het project gingen voornamelijk gebruikt worden om het verslag te schrijven en als voorbereiding voor de eindpresentatie.

\subsubsection{Uiteindelijke planning en verschillen met initiële planning}

Figuur \ref{fig:planning_initieel} in de bijlagen toont de uiteindelijke planning. In bovenstaande paragraaf werd reeds aangehaald dat Triump opgebouwd is uit meerdere kernen, in deze planning werd elk onderdeel van deze kernen uitgewerkt teneinde een gedetailleerd overzicht te geven van de gevolgde werkwijze. Het grootste verschil met de initiële planning is de toevoeging van de webinterface. De nood van een webinterface was gedurende de opstelling van de eerste planning niet gekend. Bijgevolg was het noodzakelijk de tijd nodig voor het bouwen van de webinterface te voorzien in de planning. 
De figuur toont ook aan dat er geen uitbreidingen geïmplementeerd zijn dit is voornamelijk te wijten aan de bijkomende taak en de opgelopen vertraging tijdens de ontwikkeling van core 2.

\subsubsection{Taakverdeling}

Figuur \ref{fig:planning_initieel} geeft naast de planning ook de taakverdeling weer. Meermaals per week werd er met alle leden samen gekomen om het project te evalueren en de taken te verdelen. De taken werden zo verdeeld dat elk groepslid zowel aan de frontend als aan de backend gewerkt heeft.
