
\chapter{Planning en taakverdeling}

% De initieel opgemaakte planning en de uiteindelijke uitvoering (in Ganttchart-achtige
% vorm) met een korte verklaring van de verschillen.

\subsubsection{Initiele planning}

Figuur \ref{fig:planning_initieel} in de bijlagen toont de initiële planning. De planning werd opgesteld naar aanleiding van de eerste presentatie en is bijgevolg relatief vaag. Desalniettemin is het hoofdprincipe van deze initiële planning doorheen het volledige project blijven gelden. 

Het prinicipe weergegeven in figuur \ref{fig:planning_initieel} is de gelaagde opbouw van Triump. Er wordt namelijk met drie "cores" gewerkt. De eerste van de drie cores bevat de absolute basis van de applicatie zoals communicatie met de backend, een basis-activiteit die de locatie van een gebruiker bepaald en een login en logout scherm. In de tweede core worden de basisonderdelen van Triump zoals locaties, events, rewards, groepen, gebruikers en een basis UI geïntegreerd in de applicatie. Een laatste core richt zich voornamelijk op gebruiksvriendelijkheid en afwerking. In deze core bevinden zich elementen zoals een notificatiesysteem, instellingenpagina, feedbacksysteem en een aantrijkelijke UI. Het was de bedoeling de core gelijktijdig zowel op de backend als op de frontend te implementeren.

Na het uitwerken van de kern werd tijd voorzien voor uitbreidingen. Mogelijke uitbreidingen waren: het voorzien korte spelletjes om op een tweede manier punten te kunnen verdienen, iOS of WindowsPhone versie en een integratie met Qustomer. 

De laatste weken van het project gingen voornamelijk gebruikt worden om het verslag te schrijven en om de eindpresentatie voor te bereiden.

\subsubsection{Uiteindelijke planning en verschillen met initiële planning}

Figuur \ref{fig:planning_initieel} in de bijlagen toont de uiteindelijke planning. In bovenstaande paragraaf werd de gelaagde opbouw van Triump reeds aangehaald. In deze planning werd elk onderdeel van deze lagen uitgewerkt teneinde een gedetailleerd overzicht te geven van de gevolgde werkwijze. Het grootste verschil met de initiële planning is de toevoeging van de webinterface. De nood van een webinterface was gedurende de opstelling van de eerste planning niet gekend. Bijgevolg was het noodzakelijk de tijd nodig voor het bouwen van de webinterface te voorzien in de planning. 
Uit figuur \ref{fig:planning_initieel} volgt ook dat de initieel geplande uitbreidingen niet doorgevoerd zijn. Dit is voornamelijk te wijten aan de bijkomende taak, nl. de webinterface en de langere duratie van de tweede core.

\subsubsection{Taakverdeling}

Figuur \ref{fig:planning_initieel} geeft naast de planning ook de taakverdeling weer. De taken werden zo verdeeld dat de werklast opgesplits werd over elk groepslid en dat iedereen zowel aan de frontend als aan de backend heeft gewerkt. Voor kleinere taken zoals het oplossen van een bepaalde bug of het heropmaken van een UI element werd met 'issues' gewerkt. Een 'issue' is een beschrijving van een taak of een probleem en kan toegevoegd worden aan een git repository. Vervolgens kan het groepslid die hoogst waarschijnlijk de bug geïntroduceerd heeft of het meeste ervaring heeft met een bepaalde UI-bibliotheek kan hem toekennen aan de 'issue' teneinde het probleem te verhelpen. Een voordeel van deze werkwijze is dat er een lijst opgesteld wordt van de openstaande en opgeloste problemen waardoor herorganisaties kunnen gebeuren op basis van deze informatie.

