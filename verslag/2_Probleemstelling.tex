\chapter{Probleemstelling}
Vandaag bestaan al reeds verschillende locatie-gebaseerde applicaties waarmee gebruikers hun huidige locatie kunnen delen met hun vrienden. De bekendste voorbeelden hiervan zijn Foursquare en Swarm. Foursquare is een sociale netwerksite die gebruikers plaatsen laat categoriseren en beoordelen. Het is een mobiele applicatie waarmee men hippe locaties met elkaar kan delen. Met Swarm daarentegen kunnen gebruikers 'inchecken' op plaatsen en hun locatie delen met andere gebruikers. Swarm is een mobiele applicatie, die gebruik maakt van de Foursquare's locatiedatabase. Swarm is eerder gericht op het delen van locatieupdates in plaats van ervaringen en tips zoals het geval bij Foursquare.
Hoewel Swarm en Foursquare beide gericht zijn op het sociaal gebeuren, wordt er geen nadruk gelegd op het 'samenbrengen' van mensen. 
Daarnaast onderzochten we ook online marketing platformen voor lokale ondernemingen zoals 'Qustomer'. Qustomer biedt gebruikers een klantenkaart (in de vorm van een applicatie op de mobiele telefoon) die door verschillende ondernemingen gebruikt kan worden. Net zoals bij traditionele klantenkaarten, kan een gebruiker punten opsparen en hiermee promoties verkrijgen. Qustomer en andere van deze platformen staan los van enige vorm van sociale media, en bijgevolg is de aantrekkingskracht naar gebruikers toe eerder beperkt.
Vandaag bestaan beide media, locatie-gebaseerde sociale applicaties en online marketing tools, volledig onafhankelijk van elkaar en zijn er geen toepassingen die beide concepten combineren. 
Promoties vormen een stimulans om mensen gebruik te laten maken van de diensten van een bedrijf. Indien de promoties gericht zijn op groepen mensen en volgens het Groupon-principe werken (waarbij men voordeel behaalt door in grote groep diensten aan te kopen), kunnen promoties een middel zijn om mensen samen te brengen.
Foursquare is een sociaal platform met het potentieel om een enorm publiek te bereiken wegens zijn functie als sociaal platform.
Indien men concepten zoals Foursquare en Qustomer combineert, krijgt men een platform waarbij men mensen stimuleert om samen te komen (omwille van promoties) en men deze bereikt via Foursquare.
Triump is een mobiele applicatie met als doel beide te koppelen. 
\begin{figure}[H]
	\centering
	\includegraphics[scale=0.22]{swarm_qustomer}
	\label{fig:probleemstelling}
	
\end{figure}


