\chapter{Inleiding}

Sinds de opgang van de smartphone in 2006 zijn reeds 1.2 miljard toestellen verkocht\cite{smartphone_sales}. 28 \% van het gebruik van smartphones wordt gelinkt aan sociale media. Locatie-gebaseerde applicaties zoals Foursquare\cite{foursquare} en Swarm\cite{swarm} zijn een nieuwe vorm van sociale media, met smartphones als platform, waarbij gebruikers hun locatie kunnen delen met hun vrienden.
Sociale media trekken een enorm publiek aan en deze vormen een ideale doelgroep voor online marketing. 
In dit vakoverschrijdend project wordt een toepassing ontwikkeld die ernaar streeft om sociale media te koppelen met lokale bedrijven. 
In de volgende sectie gaan we de probleemstelling verder uitdiepen, waarna we in het derde onderdeel  daaruit een concrete doelstelling naar voor brengen met functionele en non-functionele vereisten. Vervolgens wordt besproken hoe het project technisch opgebouwd is, zowel op vlak van architectuur als specifieke implementatie. In de sectie resultaten wordt de eigenlijke applicatie besproken, en teruggekeken naar de functionele en non-functionele vereisten. Hierna wordt een overzicht gegeven van de planning en samenwerking. In de laatste sectie van het verslag worden de openstaande problemen geanalyseerd.