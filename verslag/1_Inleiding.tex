\chapter{Inleiding}
%Een inleiding waarin het project geschetst wordt in een algemeen kader
Dit project kadert in het Vakoverschrijdend Project van het derde Bachelorjaar Burgelijk Ingenieur Computerwetenschappen. Studenten krijgen hierbij een aantal voorstellen aangeboden vanuit de verschillende vakgroepen, maar zijn ook vrij om zelf een projectvoorstel te doen. Gedurende 12 weken wordt vervolgens in groepen van 4 studenten gewerkt, waarbij een efficiënte en doelmatige aanpak van het project verwacht wordt. In dit verslag wordt een bespreking gegeven van het project Triump. Hierbij wordt zowel aandacht besteed aan de technische onderdelen als de projectmatige aspecten van de samenwerking.


Triump is een locatie-gebaseerde applicatie, sterk geïntegreerd met bestaande sociale media.
Sociale media worden in steeds meer aspecten van ons dagelijks leven moeiteloos geïntegreerd. Van het delen van statussen via Facebook, tot het delen van foto’s via Instagram, voor zo goed als elke dagelijkse activiteit bestaat wel een online platform dat inspeelt op het sociaal aspect behorend bij die activiteit.
Triump is een Android applicatie bedacht en ontworpen door studenten. Het doel is een overzichtelijke en gebruiksvriendelijke applicatie te ontwikkelen die verschillende sociale media in 1 applicatie bundelt. 
Met Triump kunnen gebruikers hun locatie met hun vrienden delen, lid worden van groepen, deelnemen aan evenementen en promoties winnen.
Triump is een vernieuwende manier om mensen dichter bij elkaar te brengen, door ze een gemeenschappelijk doel, namelijk evenementen, en bijhorende promoties voor de ogen te houden.

\begin{figure}[H]
	\centering
	\includegraphics[scale=0.175]{ready_set_triump}
	\label{fig:inleiding}
	
\end{figure}
