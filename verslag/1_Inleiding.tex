\chapter{Inleiding}

Sinds de opgang van smartphones in 2006 zijn reeds 1.2 miljard toestellen verkocht\cite{smartphone_sales}. 28 \% van het gebruik van smartphones wordt vandaag gelinkt aan sociale media. Locatie-gebaseerde applicaties zoals Foursquare\cite{foursquare} en Swarm\cite{swarm} zijn een nieuwe vorm van sociale media, voortstuwend op de opmars van smartphones, waarbij gebruikers hun locatie kunnen delen met hun vrienden.
Sociale media trekken een enorm publiek aan en vormen zo een ideale doelgroep voor online marketing. 

In dit vakoverschrijdend project wordt een toepassing ontwikkeld die ernaar streeft om sociale media te integreren met lokale bedrijven.
De applicatie vormt voor gebruikers een sociaal platform, terwijl bedrijven het als online marketing tool kunnen inzetten.

In de volgende sectie gaan we de probleemstelling verder uitdiepen. In het derde onderdeel  gaan we dan daaruit een concrete doelstelling naar voor brengen met functionele en niet-functionele vereisten. Vervolgens wordt besproken hoe het project technisch opgebouwd is, zowel op vlak van architectuur als specifieke implementatie. In de sectie resultaten wordt de eigenlijke applicatie besproken, en teruggekeken naar de functionele en niet-functionele vereisten. Hierna wordt een overzicht gegeven van de planning en samenwerking gedurende het project. In de laatste sectie van het verslag worden de openstaande problemen geanalyseerd.